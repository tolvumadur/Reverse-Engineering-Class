\vskip.25in
\noindent\textbf{Textbook:}
\vskip.25in
The textbook for this course is required for the readings and projects. 
If you are around somebody in the class often, you could probably get away with sharing a copy.
No access code is required to access the assignment materials. 
\vskip.25in
\begin{centering}
\textbf{Practical Malware Analysis: The Hands-On Guide to Dissecting Malicious Software} by Michael Sikorski and Andrew Honig
\end{centering}
\vskip.25in
\textit{Why do we need a textbook?} This book is well-known in the reverse engineering community. 
Having worked through its exercises will give you instant credibility with potential employers.
The included assignments are high-quality, and specifically designed to show you each concept we are learning.

\vskip.25in
\noindent\textbf{Optional Secondary Resource:}

Learning Malware Analysis by Monnappa K. A. (Published June 2018) ISBN: 9781788392501

\vskip.25in
\noindent\textbf{Hardware Requirements:}
\vskip.25in
This course will involve analyzing Windows 10 malware, which means you will need a machine capable of simultaneously virtualizing Windows 10 and a *nix OS.
\vskip.25in
We recommend the following specifications at minimum:
\begin{itemize}
    \item 50+ GB HDD or (USB3.0+ external)
    \item 8GB+ RAM  
    \item 1GHz+, x86-64 dual core or better processor supporting virtualization (not ARM)
    \item a USB 3.0+ port
\end{itemize}

You will need to be able to run a *nix OS on your hardware -- this may mean dual booting if you usually run Windows (WSL will not be sufficient).

