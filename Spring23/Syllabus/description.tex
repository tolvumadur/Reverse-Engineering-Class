\vskip.25in
\noindent\textbf{Course Summary}:
Reverse engineering malware is a valuable skill to protect society from cyber threats, and is a lucrative career option for a computer scientist.
This challenging course will prepare students to begin a career in malware analysis by building a portfolio of reverse engineering projects.
Students will gain experience with reverse engineering techniques, tools, and concepts that will allow them to step into an entry-level malware analyst position.

\vskip.25in
\noindent\textbf{Learning Outcomes}:
\begin{itemize}
    \item Students will learn how malware behaves, spreads, and is controlled.
    \item Students will learn how to safely analyze malware in controlled environments.
    \item Students will learn how malware seeks to hide in systems.
    \item Students will learn to perform static analysis of binaries using simple tools
    \item Students will learn how malware obfuscates itself to avoid analysis, including using crypto packers, polymorphism, and sandbox detection
    \item Students will learn to perform decompilation and control-flow analysis of binaries using Ghidra
    \item Students will learn to dynamically analyze malware in a sandbox environment while observing network traffic, resource consumption, and system calls
    \item Students will learn to detect malware running with operating-system level permissions (rootkits)
    \item Students will learn memory forensics techniques to detect malware hidden within benign processes
\end{itemize}